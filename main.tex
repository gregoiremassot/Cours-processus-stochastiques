\documentclass{article}
\usepackage[utf8]{inputenc}
\usepackage[T1]{fontenc}

\usepackage{amsthm}
\usepackage{amsmath}
\usepackage{amssymb}
\usepackage{mathrsfs}

\title{Cours processus stochastiques}
\author{Anne Loisel}
\date{Octobre 2018}

\begin{document}

\maketitle
\chapter{Martingales et temps d'arrêt}
\section{Généralités sur les processus stochastiques}
On commence par introduire une notion mathématique fondamentale qui joue un rôle centrale en théorie des marchés financiers
\newline
\newline
Soit $(\Omega, \mathscr{F}, \mathbb{R})$ un espace probabilisé et $\{\mathscr{F}_{t}\}$ une famille de sous-tribus de $\mathscr{F}$. On dit que ${\mathscr{F}_{t}}$ est une filtration si c'est une famille croissante au sens où
\[\mathscr{F}_{S} \subset \mathscr{F}_{T}, \forall S \leq T \]

Remarques :
Il faut comprendre $\mathscr{F}_{T}$ comme "l'information au temps t", plus le temps croit $s \leq t$ plus on a d'information
Une filtration 

\end{document}
